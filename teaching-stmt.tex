\documentclass[12pt]{article}
\usepackage{amsmath}
\usepackage{amssymb}
\usepackage{amsthm}
\usepackage{amscd}
\usepackage{amsfonts}
\usepackage{graphicx}%
\usepackage{fancyhdr}


\theoremstyle{plain} \numberwithin{equation}{section}
\newtheorem{theorem}{Theorem}[section]
\newtheorem{corollary}[theorem]{Corollary}
\newtheorem{conjecture}{Conjecture}
\newtheorem{lemma}[theorem]{Lemma}
\newtheorem{proposition}[theorem]{Proposition}
\theoremstyle{definition}
\newtheorem{definition}[theorem]{Definition}
\newtheorem{finalremark}[theorem]{Final Remark}
\newtheorem{remark}[theorem]{Remark}
\newtheorem{example}[theorem]{Example}
\newtheorem{question}{Question} \topmargin-2cm

%\newcommand{\kl}[1]{\marginpar{\sc KL}{\bf #1}}
\newcommand{\kl}[1]{}

\newcommand{\comment}[1]{}
\textwidth6in

\setlength{\topmargin}{0in} \addtolength{\topmargin}{-\headheight}
\addtolength{\topmargin}{-\headsep}

\setlength{\oddsidemargin}{0in}

\oddsidemargin  0.0in \evensidemargin 0.0in \parindent0em

\pagestyle{fancy}\lhead{Teaching Statement} \rhead{September 2019}
\chead{{\large{\bf Kenny Zhuo Ming Lu}}} \lfoot{} \rfoot{\bf \thepage} \cfoot{}

\newcounter{list}

\begin{document}

\raisebox{1cm}

\section{Goal of Teaching}
My main goal of teaching is to
\begin{itemize}
   \item lead and curate students into various computer science subjects,
   \item equip the students with the state-of-art technical knowledge
     and skills,
   \item assist the students to develop computational thinking and problem solving skills. 
\end{itemize}
%
The following are some of my personal believes towards teaching

\section{Learning through examples}
I believe that the best teaching model is to learn through examples.
This is one of the key to cognitive learning. Most of the students will
feel lost if the lecturer bring up the mathematical formula/concept abstraction/theorem
at the beginning of a class. It is the best practice to draw the attention
from the audience by giving some simple and interesting examples, to motivate them with 
the importance of the concepts and theories behind these examples. For instance, in
the module object oriented programming, my students have already had
some basic programming background knowledge such as variables,
sequence statements, conditional statements, loop statements and
functions. I often would first start with an example from which variables, functions and similar routine are
repeated unnecessarily in the absence of classes and objects. It is
followed by a careful demonstration of how object oriented concept can be applied
to fix these issues. With the motivating examples, students will
understand the technical details a lot better.

\section{Contextualizing the examples}
Examples make learning more relateable. 
I prefer contextualized examples which are in general
more effective. In the module object oriented programming, students
use a text book with many examples. However a large number of
those examples were set in the context of US which are not so
relatable to our post secondary students who are based
in Singapore. For instance, there is this example making an analogy
between the concept of encapsulation and cars. In Singapore, students
at that age between 17 to 19 have no or little
experience of cars and driving. To address the issue, I prefer substituting these
examples with some altneratives which are relevant to the students,
e.g. using laptops and computer mice, etc. Students find these examples are
helpful compared to those mentioned in the textbooks.


\section{Learner-initiated style}
As a tertiary level teacher, I believe that the best knowledge we can offer
to the students is the ability of self-learning. 
A perfect education system should encourage creativity and independent 
thinking. The best learning outcome is that the students will be able to construct 
new ideas and solves new problems independently. If circumstances
permit, I adopt learner-initiated style teaching in class to cultivate
self-directed learning ability. For instance, during the delivery of
the module of parallel programming of big data, I adopted a flipped
classroom approach. I pre-recorded all the lectures in the form of
a sequence of interactive, bite-size video clips. Students are encouraged to go through the
lecture materials before attending the actual classes.
I find this strategy is effective for matured learners who prefer
learning at their own pace.
We gained productivity during the face-to-face class sessions by discussing
topics and difficult concepts in-depth. Students also prefer pre-recorded lectures because of
the convenience for exam revision.


\section{Differentiate the learners}
Every student is unique, so is his/her learning style. While
it is not feasible to personalize
the learning content for every student, we try our best to differentiate
the learners whenever possible. For instance, through self-initiated,
self-paced style learning, I encourage the students to take their own
time in grasping the parts that they are unsure of. Even in terms of
progress check and assessment, I try to give slight different
assessment criteria based on the strength and weakness of the students
without sacrificing fairness. For instance, I allow students to define
partially their own learning goals and graded based on the goals they
set. Through these exercises, I find that the students, in particular
the younger / teenager ones, could be motivated better.


\section{Train the minds by freeing the eyes and the hands}
As a computer scientist and programmer, I found the best enviroment
of tackling a problem is staying away from the computer screens. I
also found that students who are overly reliant on the computer and
interactive development environment often under-perform during the
exams. During the class, in particular when the abstract concepts are
taught, I tend to developthe example, theory, solution or proof on
the whiteboard by writing them down. It might be slower, but it allows 
the students to follow the development of the
solutions instead of memorizing the answers. Instead of being busy with copying
the solutions, the students are given the time to pause and to think about the
process. Besides utilsing the white boards, I also actively develop
some class activities that does not invovle IT infrastructure. For
instance, when teaching students the concept of MapReduce, we organize
a team based activity using poker cards. The goal of the activity is to parallelize the
sorting process of the poker decks using  map and reduce. 
Students are divided into groups. Within a group, students are assigned with roles of
mappers and reducers. Instructions for the mappers and reducers are
given to the students. Students will execute their sorting process by
following the given instructions based on their
roles. Through the activities and the reflection the students understand
how map reduce works and its pros and cons. 

\section{Conclusion}
Teaching is re-learning. Through teaching, I inspired some
students and was inspired by them. Through teaching, I am able to look at
the same problem from multiple angles, which broadened my view and deepens my
understanding. I love and enjoy teaching. 



\comment{
\section{My Belief Towards Teaching}
{\em  
``Give a man a fish and you feed him for a day. Teach him how to 
fish and you feed him for a lifetime.''
} \\ \\
{\em
$~~~~~~~~~~~~~~~~~~~~~~~~~~~~~~~~~~~~~~~~~~~~~~~~~~~~~~~~~~~~~~~~~
~~~~~~~~~~~~~~~~~~~~~~~~$ - Lao-Tzu
}
\\ \\
As a teacher, I strongly believe that the best knowledge we can offer
to the students is the ability of self-learning. 
A perfect education system should encourage creativity and independent 
thinking, so that the students will be able to construct 
new ideas and solves new problems independently.



I see teaching not as a monolog. A class requires interaction. To stimulate 
responses from the students, I always prepare the class materials 
carefully. Part of the materials consist of ideas which have been
understood or can be easily understood by the students. 
It is a bad idea to give a lecture that makes the students feel that
they are incapable to learn.
Whilst the other part consist of some challenging problems/ideas,
as to encourage creative thinking.


\section{My Teaching Experience and Expertise}
During my graduate study in National University of Signapore,
I actively particated in many teaching activities, which are elaborated as follows.

In the academic year 2001/2002, I was tutoring 
an undergraduate course {\em Introduction to Database}. 

In the academic year 2002/2003/2004, I was a project advisor for the undergraduate course
{\em Information Systems Development Project}.

In the academic year 2007/2008, I was a teaching assitance of a graduate course
{\em Compiler Design}.

Since my research focus in prgoramming language, my teaching is focus in 
software engineering and programming language related subjects.
In future, I would like to develop the following courses.

\begin{itemize}
 \item {\em Introduction to programming languages and algorithms:} 
This course is an introduction to the basic methodology of various modern programming 
languages such as Java and C\#. We will study the core of the object oriented 
languages. Then we will gradually expand the core language to cover 
some common libraries. We will study different types of data structures and algorithms
which we come across in the software development projects.
 \item {\em Advanced topics in programming languages and software development}
In this course, we will focus in the advanced features in programming languages.
E.g. advanced type systems, generic programming, data-stream programming,
parallel programming and distributive programming. We will also study 
the pratical aspects of these programming paradigm in the real world software development.
\end{itemize}
}
\end{document}
