\documentclass[12pt]{article}
\usepackage{amsmath}
\usepackage{amssymb}
\usepackage{amsthm}
\usepackage{amscd}
\usepackage{amsfonts}
\usepackage{graphicx}%
\usepackage{fancyhdr}


\theoremstyle{plain} \numberwithin{equation}{section}
\newtheorem{theorem}{Theorem}[section]
\newtheorem{corollary}[theorem]{Corollary}
\newtheorem{conjecture}{Conjecture}
\newtheorem{lemma}[theorem]{Lemma}
\newtheorem{proposition}[theorem]{Proposition}
\theoremstyle{definition}
\newtheorem{definition}[theorem]{Definition}
\newtheorem{finalremark}[theorem]{Final Remark}
\newtheorem{remark}[theorem]{Remark}
\newtheorem{example}[theorem]{Example}
\newtheorem{question}{Question} \topmargin-2cm

%\newcommand{\kl}[1]{\marginpar{\sc KL}{\bf #1}}
\newcommand{\kl}[1]{}

\newcommand{\comment}[1]{}
\textwidth6in

\setlength{\topmargin}{0in} \addtolength{\topmargin}{-\headheight}
\addtolength{\topmargin}{-\headsep}

\setlength{\oddsidemargin}{0in}

\oddsidemargin  0.0in \evensidemargin 0.0in \parindent0em

\pagestyle{fancy}\lhead{Teaching Statement} \rhead{March 2012}
\chead{{\large{\bf Kenny Zhuo Ming Lu}}} \lfoot{} \rfoot{\bf \thepage} \cfoot{}

\newcounter{list}

\begin{document}

\raisebox{1cm}

\section{The Goal}
My basic goal of teaching is to educate the students about
various computer programming theories, equip the students with 
the state-of-art programming techniques and concepts.
My high level objective is to help the students to build up
their problem solving skill. 
This skill set is essential because it is applicable not only to 
IT industry but also to other industries and our daily life. 
\\ \\

\section{My Belief Towards Teaching}
{\em  
``Give a man a fish and you feed him for a day. Teach him how to 
fish and you feed him for a lifetime.''
} \\ \\
{\em
$~~~~~~~~~~~~~~~~~~~~~~~~~~~~~~~~~~~~~~~~~~~~~~~~~~~~~~~~~~~~~~~~~
~~~~~~~~~~~~~~~~~~~~~~~~$ - Lao-Tzu
}
\\ \\
As a teacher, I strongly believe that the best knowledge we can offer
to the students is the ability of self-learning. 
A perfect education system should encourage creativity and independent 
thinking, so that the students will be able to construct 
new ideas and solves new problems independently.


I also believe that the best teaching model is to go by examples.
This is one of the key to cognitive learning. Most of the students will
feel lost if the lecturer bring up the mathematical formula
at the beginning of a class. It is the best pratice to draw the attention
from the audience by giving simple and interesting examples, motivate them with 
the importance of the concept and the theory behind.

I see teaching not as a monolog. A class requires interaction. To stimulate 
responses from the students, I always prepare the class materials 
carefully. Part of the materials consist of ideas which have been
understood or can be easily understood by the students. 
It is a bad idea to give a lecture that makes the students feel that
they are incapable to learn.
Whilst the other part consist of some challenging problems/ideas,
as to encourage creative thinking.


\section{My Teaching Experience and Expertise}
During my graduate study in National University of Signapore,
I actively particated in many teaching activities, which are elaborated as follows.

In the academic year 2001/2002, I was tutoring 
an undergraduate course {\em Introduction to Database}. 

In the academic year 2002/2003/2004, I was a project advisor for the undergraduate course
{\em Information Systems Development Project}.

In the academic year 2007/2008, I was a teaching assitance of a graduate course
{\em Compiler Design}.

Since my research focus in prgoramming language, my teaching is focus in 
software engineering and programming language related subjects.
In future, I would like to develop the following courses.

\begin{itemize}
 \item {\em Introduction to programming languages and algorithms:} 
This course is an introduction to the basic methodology of various modern programming 
languages such as Java and C\#. We will study the core of the object oriented 
languages. Then we will gradually expand the core language to cover 
some common libraries. We will study different types of data structures and algorithms
which we come across in the software development projects.
 \item {\em Advanced topics in programming languages and software development}
In this course, we will focus in the advanced features in programming languages.
E.g. advanced type systems, generic programming, data-stream programming,
parallel programming and distributive programming. We will also study 
the pratical aspects of these programming paradigm in the real world software development.
\end{itemize}
\end{document}