\documentclass[12pt]{article}
\usepackage{amsmath}
\usepackage{amssymb}
\usepackage{amsthm}
\usepackage{amscd}
\usepackage{amsfonts}
\usepackage{graphicx}%
\usepackage{fancyhdr}


\theoremstyle{plain} \numberwithin{equation}{section}
\newtheorem{theorem}{Theorem}[section]
\newtheorem{corollary}[theorem]{Corollary}
\newtheorem{conjecture}{Conjecture}
\newtheorem{lemma}[theorem]{Lemma}
\newtheorem{proposition}[theorem]{Proposition}
\theoremstyle{definition}
\newtheorem{definition}[theorem]{Definition}
\newtheorem{finalremark}[theorem]{Final Remark}
\newtheorem{remark}[theorem]{Remark}
\newtheorem{example}[theorem]{Example}
\newtheorem{question}{Question} \topmargin-2cm

%\newcommand{\kl}[1]{\marginpar{\sc KL}{\bf #1}}
\newcommand{\kl}[1]{}

\newcommand{\comment}[1]{}
\textwidth6in

\setlength{\topmargin}{0in} \addtolength{\topmargin}{-\headheight}
\addtolength{\topmargin}{-\headsep}

\setlength{\oddsidemargin}{0in}

\oddsidemargin  0.0in \evensidemargin 0.0in \parindent0em

\pagestyle{fancy}\lhead{Research Statement} \rhead{July 2019}
\chead{{\large{\bf Kenny Zhuo Ming Lu}}} \lfoot{} \rfoot{\bf \thepage} \cfoot{}

\newcounter{list}

\begin{document}

\raisebox{1cm}



\section{Motivation}

\kl{What is the goal of my research?}
The goal of my research is to enhance the usability, reliability, 
maintainability and performance of the software system via programming language
technology. Many modern programming languages evolve over time.
The need of new programming tools and language features is 
emerging due to new system development platforms and new 
business requirements. A good programming language and model must equip
programmers with ``state-of-the-art'' language features
and libraries, so that they can focus on the actual problem without
worrying about the underlying lower-level routines.


\kl{What is the focus?}
In pursuit of this goal, my research focuses on
{\em Type Systems},  {\em Formal Languages and Methods} and {\em Software Analysis}.
I am passionate in exploring findings and results in these domains and apply them into real world applications
such as software engineering, data engineering and software security projects.


\section{Past and on-going ressearch}

\subsection{XHaskell}
XHaskell is a language extension to Haskell \cite{xhaskell} 
which combines regular expression type, 
which is available in XDuce \cite{XDuceTyPHD}, and parametric
polymorphism and algebraic datatypes, which are available in Haskell.
Parametric polymorphism is a very common features in most of the 
modern languages such as Java and Haskell. It allows us to specify a
common routine for different types of values. Algebraic datatype
allows us to build data structures in Haskell. Regular expression
type, on the other hand, is an unique language feature. 
It is originated in the XDuce language. It is used to capture
static information of semi-structured data, such as XML. In
combination with regular expression pattern and semantic subtyping,
programmers are able to specify XML transformation routine concisely.
The XDuce type system provides static guarantees for the program, so that
the program will not running into error when being executed. 

As part of my PhD thesis, I have developed a
type-directed translation scheme which translates XHaskell programs
(Haskell code + XDuce code) into pure Haskell programs. Therefore
the resulting Haskell programs can be compiled into binary executables.
There are two major advantages. First, XDuce programmers have full
access to the existing libraries provided by Haskell; at the same
time Haskell programmers are able to describe powerful transformation
using regular expression types and patterns. Secondly, the XHaskell
programs are much more efficient as compared their XDuce counter-parts.
There are two factors contributing to the performance gain. 
The first factor is the type-based analysis which is performed on the 
source programs. The pattern matching routines are highly optimized
as they are represented in terms of type-specific coercions. The second 
factor is that XHaskell programs are compiled into binary code whereas
XDuce programs are always interpreted. 

The main technique being used in this project is the
``proofs-are-programs'' principle (A.K.A. Curry-Howard Isomorphism). 
This technique is not only applicable  to functional
languages, but also to imperative language such as C and Java.

We proposed the first prototype implementation using Haskell type class
 \cite{semantic-subtyping}. In a later work \cite{ml-workshop05}, 
we reformulated the prototype into a language extension. 
In another publication \cite{ifl2007}, we presented the complete implementation of the system.


\subsection{Regular Expression, Derivatives, Partial Derivatives and unambiguity}

Besides as an extension to type systems, regular expression is widely adopted in many
real world application as a simple Domain Specific Language for pattern matching and data processing.
Despites its populairity, a lot of software bugs and security loop holes arising due to the
lack of proper debugging support and un-verified library implementation.

\kl{debugging}



\kl{rewriting frontier}

{\em statically-typed
programming languages}, such as C and Java. These languages
provide specific compile-time guarantee about the absence of certain
program errors. As a result, new language features will not create new
burden to the programmer, as they must be asserted by the type system.
Further more, the static property of a program often allows us 
to perform compile-time optimization. 

\kl{What is the problem?}
Unfortunately, the evolution of the programming languages 
is not catching up with the pace of hardware evolution.
For example, we have stepped into a multi-core era.
But many of us are still worried about whether our programs will terminate when they are executed on a multi-core system, until we realize
that {\em software transactional memory} \cite{stm} may automatically 
resolve the potential deadlock for us. 
My research explores the possibility of employing the cutting-edge
programming idioms and techniques into the existing 
programming languages, enhancing the usability and performance
in the context of the new development platforms and new requirement. 
Below I describe what I have achieved as well as my future directions.


\section{Type-based Language Extension and Optimization}

The current focus of my research is {\em type-based language 
extension and optimization}. As part of my 
graduate project, I co-authored a language extension called XHaskell.


%\subsection{Multi-headed Erlang}




\section{Future Directions}

\subsection{Type-directed Program Parallelization}
Program Parallelization has become popular recently. There have been 
some attempts to automate the process of ``parallelizing'' a
sequential program.

One nice property about pure functional languages is that
the variables are immutable. Therefore it is easy for us to identify
which program snippet is independent of the rest of the program
and therefore we can execute that snippet in parallel with the rest of
the code. 

There have been some research projects trying to apply this concept
to the context of imperative languages such as C++ \cite{mapreduce} 
and Java \cite{hadoop}. However, there are limitations with these
approaches. The programmer needs to recast the existing problem into 
one which is solvable by using the {\tt map} and {\tt reduce} operations.
This process is not automated, and very often it is non-trivial.

With static program analysis, the compiler
is able to identify program patterns which can be specialized.
With type-directed transformation, we are able to rewrite the program 
automatically into a new form such that it is parallelizable. The 
type soundness property guarantees that the resulting program is
well-typed. Through static analysis, it is easy for us to justify that the
transformation preserves the original semantics.


\subsection{Distributed Program Synchronization}
Distributed Computing is another interesting topic. In the distributed
computing setting, program and data are fragmented into smaller parts and 
distributed to many nodes across the network, then executed concurrently. 
In the absence of shared
memory, process synchronization becomes essential. Traditional process
synchronization methods are often too low-level. The message passing routines
are deeply coupled within the program logic. This greatly impacts 
the flexibility of design and the maintainability of the codes.

Recently the Erlang \cite{erlang} style of actor programming is becoming
an interesting case study. In Erlang, inter-process communication
routines are expressed in terms of explicit {\tt send} and {\tt receive}
constructs. Data and messages are passed among nodes via these
constructs. Incorporating this feature into main stream imperative
languages such as C and Java becomes an exciting problem. 

Software Transactional Memory (STM) \cite{stm} was recently introduced to
community. The purpose of using STM is to reduce the effort of 
writing deadlock-free programs. The programmer is freed from 
writing explicit locks and semaphores for synchronization. (This is
similar to the introduction of garbage collection, which 
frees the programmer from writing {\tt malloc} and {\tt free} statements.) Currrently, Software Transactional Memory is only applicable in the context of 
shared memory processors (SMP). It is also possible to apply the
idea to distributed computing settings. For instance, via memcache
we are able to allow multiple nodes on a network to share a common
memory space.  

My idea is to unify the concepts from these two approaches into a
calculus and apply it to main stream languages such as C and Java. 
This would allow us to develop a highly-useful distributed computing framework.
\comment{
Recently, there have been some work \cite{cchr} which unifies the ideas of the 
above two projects. Via a notion of Concurrent Constraint Handling
Rules, it is possible to describe these two features in one single 
calculus. Therefore, there is ample of space we can explore the hybrid
approach of the two.
}






\newpage
\bibliographystyle{plain}
\bibliography{main}

\end{document}
