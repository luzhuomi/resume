\documentclass[12pt]{article}
\usepackage{amsmath}
\usepackage{amssymb}
\usepackage{amsthm}
\usepackage{amscd}
\usepackage{amsfonts}
\usepackage{graphicx}%
\usepackage{fancyhdr}


\theoremstyle{plain} \numberwithin{equation}{section}
\newtheorem{theorem}{Theorem}[section]
\newtheorem{corollary}[theorem]{Corollary}
\newtheorem{conjecture}{Conjecture}
\newtheorem{lemma}[theorem]{Lemma}
\newtheorem{proposition}[theorem]{Proposition}
\theoremstyle{definition}
\newtheorem{definition}[theorem]{Definition}
\newtheorem{finalremark}[theorem]{Final Remark}
\newtheorem{remark}[theorem]{Remark}
\newtheorem{example}[theorem]{Example}
\newtheorem{question}{Question} \topmargin-2cm

%\newcommand{\kl}[1]{\marginpar{\sc KL}{\bf #1}}
\newcommand{\kl}[1]{}

\newcommand{\comment}[1]{}
\textwidth6in

\setlength{\topmargin}{0in} \addtolength{\topmargin}{-\headheight}
\addtolength{\topmargin}{-\headsep}

\setlength{\oddsidemargin}{0in}

\oddsidemargin  0.0in \evensidemargin 0.0in \parindent0em

\pagestyle{fancy}\lhead{Research Statement} \rhead{July 2019}
\chead{{\large{\bf Kenny Zhuo Ming Lu}}} \lfoot{} \rfoot{\bf \thepage} \cfoot{}

\newcounter{list}

\begin{document}

\raisebox{1cm}



\section{Motivation}

\kl{What is the goal of my research?}
The goal of my research is to enhance the usability, reliability, 
maintainability and performance of the software system via programming language
technologies. Many modern programming languages evolve over time.
The need of new programming tools, compiler and  language features is 
emerging due to new system development platforms and new 
business requirements. A good programming language and model must equip
programmers with ``state-of-the-art'' language features
and libraries, so that they can focus on the actual problems without
worrying about the underlying lower-level routines.


\kl{What is the focus?}
In pursuit of this goal, my research focuses on
{\em Type Systems},  {\em Formal Languages and Methods} and {\em Software Analysis}.
I am passionate in exploring findings and results in these domains and apply them
into real world applications,
such as software engineering, data engineering and software security projects.


\section{Past and on-going ressearch}

\subsection{Regular Expression Type and Pattern Matching}
XHaskell is a language extension to Haskell \cite{xhaskell} 
which combines regular expression type, 
which is available in XDuce \cite{XDuceTyPHD}, and parametric
polymorphism and algebraic datatypes, which are available in Haskell.
Parametric polymorphism is a very common features in most of the 
modern languages such as Java, C\# and Haskell. It allows programmers to specify a
common routine for different types of values. Algebraic datatype
allows programmers to build data structures in functional languages,
such as Scala and Haskell. Regular expression
type, on the other hand, is a unique language feature. 
It is originated in the XDuce language. It is used to capture
static information of semi-structured data, such as XML. In
combination with regular expression patterns and semantic subtyping,
programmers are able to specify semi-structured data transformation routine concisely.
The XDuce type system provides static guarantees for the program, so that
the most of the run-time errors of the program will be eliminated.

As part of my PhD thesis, I have developed a
type-directed translation scheme which translates XHaskell programs
(Haskell code + XDuce code) into pure Haskell programs. Therefore
the resulting Haskell programs can be compiled into binary executables.
There are two major advantages. First, XDuce programmers have full
access to the existing libraries provided by Haskell; at the same
time Haskell programmers are able to describe powerful transformation
using regular expression types and patterns. Secondly, the XHaskell
programs are much more efficient as compared their XDuce counter-parts.
There are two factors contributing to the performance gain. 
The first factor is the type-based analysis which is performed on the 
source programs. The pattern matching routines are highly optimized
as they are represented in terms of type-specific coercions. The second 
factor is that XHaskell programs are compiled into binary code whereas
XDuce programs are executed by interpreters.

The main technique being used in this project is the
``proofs-are-programs'' principle (A.K.A. Curry-Howard Isomorphism). 
This technique is not only applicable to functional
languages, but also to imperative language such as C and Java.

We proposed the first prototype implementation using Haskell type class
 \cite{semantic-subtyping}. In a later work \cite{ml-workshop05}, 
we reformulated the prototype as a language extension to ML.
In the publication \cite{DBLP:conf/ifl/SulzmannL07}, we presented the complete implementation of the system.


\subsection{Regular Expression, Derivatives, Partial Derivatives and
  Ambiguity Detection}

Besides as an extension to type systems, regular expression is widely used in many
applications for pattern matching and data processing.
Despites its populairity, a lot of software bugs and security loop holes arising due to the
lack of proper debugging support and un-verified library implementation. 

We proposed a new suite of regular expression algorithms and tools based on two well-studied
but long-neglected regular expression operations,
Brzozowski's derivative \cite{321249} and Antimirov's partial derivative \cite{Antimirov96Partial}.

In our work \cite{DBLP:conf/ppdp/SulzmannL12}, we proposed and developed a regular expression
matching algorithm using partial derivatives. Unlike many other existing algorithms, our
partial derivative matching algorithm does not require any explicit automata construction.
The algorithm is proven to be correct with respect to the greedy matching policies. Empircal
benchmarks show that the algorithm is efficient enought to used in real world applications.

Besides greedy matching policy (which is a local longest matching approach),
POSIX matching policy (which is a global longest matching approach) is also being studied
and used in some applications. Many existing implementation are un-verified and buggy.
In our work \cite{DBLP:conf/flops/SulzmannL14}, 
we show how to obtain a POSIX algorithm for the general parsing problem based on Brzozowski’s regular expression derivatives.

Derivatives can be also applied in detecting ambiguous regular
expressions. A regular expression is ambiguous if there exist two
different ways of parsing the same word. In a latter work of ours
\cite{DBLP:conf/wia/SulzmannL16}, we present a novel method
based on derivatives to aid the user in diagnosing ambiguous regular expressions.


In a recent work \cite{DBLP:journals/corr/abs-1908-03710}, we
illustrated that in combination with Arden's Law, regular expression derivatives can be used in solving
regular equations. In this work, we also identified some conditions
under which the solution is unambiguous. The results can be applied
into at least two applications. Firstly, we can convert a DFA to  a
regular expression which is unambiguous. Secondly, intersection,
substraction, shuffling operations among regular expressions can be
solved symbolically without the explicit conversion to DFAs.



\subsection{Software Security and Code Obfuscation}

We look into the problems of code obfuscation. In our work
\cite{DBLP:conf/pepm/Lu19},
We propose a new control flow obfuscation technique by rewriting the
source program in the continuation passing style (CPS).
The continuation is encoded through higher order combinators and
function pointers at the target language level.
As a result, the original control flow graph is fragmented which makes
any software tampering attempt through binary static analysis hard. 



\section{Work in Progress, Future Direction}

\subsection{Regular Expression Diagnosis Continued}
The default regular expression libraries in main stream programming
languages such as Java, C\# and Python are based on backtracking. Besides
the performance overheads, backtracking based implementation imposed
security loopholes for possible attack \cite{ReDoS} and software bugs
\cite{JDK-5050507}. We believe that some of the works we have
completed in umabiguous diagnosis could be transferred to detect for
evil regular expressions.

Besides detection, we are interested into debugging and fix
suggestion. For instance, in the event of an ambiguous or an evil
regular expression is detected, it would be nice that there a way to
identify a sound and comprehensible fix. We conjecture that the 
the symbolic approaches will have certain advantages over DFA
construction approaches.


\subsection{Software Obfuscation}
We are working towards a formal proof through which we can verify the
competence level of the CPS based code obfuscation technique that we
developed in work \cite{DBLP:conf/pepm/Lu19}.  Concurrently, we are
looking into possible extensions in multiple dimensions. For instance,
we are considering utilizing exception handling. We could look into
the possibility of byte code level obfuscation, since many modern
compilers use Static Single Assignment (SSA) as the intermediate
language. Our results can be easily adapted to the byte code level.
Lastly, we would like to explore the possibilty of using deliminated
continuations. 


\subsection{Software Analysis}

\kl{debugging}



\kl{rewriting frontier}
\comment{
{\em statically-typed
programming languages}, such as C and Java. These languages
provide specific compile-time guarantee about the absence of certain
program errors. As a result, new language features will not create new
burden to the programmer, as they must be asserted by the type system.
Further more, the static property of a program often allows us 
to perform compile-time optimization. 

\kl{What is the problem?}
Unfortunately, the evolution of the programming languages 
is not catching up with the pace of hardware evolution.
For example, we have stepped into a multi-core era.
But many of us are still worried about whether our programs will terminate when they are executed on a multi-core system, until we realize
that {\em software transactional memory} \cite{stm} may automatically 
resolve the potential deadlock for us. 
My research explores the possibility of employing the cutting-edge
programming idioms and techniques into the existing 
programming languages, enhancing the usability and performance
in the context of the new development platforms and new requirement. 
Below I describe what I have achieved as well as my future directions.


\section{Type-based Language Extension and Optimization}

The current focus of my research is {\em type-based language 
extension and optimization}. As part of my 
graduate project, I co-authored a language extension called XHaskell.


%\subsection{Multi-headed Erlang}




\section{Future Directions}

\subsection{Type-directed Program Parallelization}
Program Parallelization has become popular recently. There have been 
some attempts to automate the process of ``parallelizing'' a
sequential program.

One nice property about pure functional languages is that
the variables are immutable. Therefore it is easy for us to identify
which program snippet is independent of the rest of the program
and therefore we can execute that snippet in parallel with the rest of
the code. 

There have been some research projects trying to apply this concept
to the context of imperative languages such as C++ \cite{mapreduce} 
and Java \cite{hadoop}. However, there are limitations with these
approaches. The programmer needs to recast the existing problem into 
one which is solvable by using the {\tt map} and {\tt reduce} operations.
This process is not automated, and very often it is non-trivial.

With static program analysis, the compiler
is able to identify program patterns which can be specialized.
With type-directed transformation, we are able to rewrite the program 
automatically into a new form such that it is parallelizable. The 
type soundness property guarantees that the resulting program is
well-typed. Through static analysis, it is easy for us to justify that the
transformation preserves the original semantics.


\subsection{Distributed Program Synchronization}
Distributed Computing is another interesting topic. In the distributed
computing setting, program and data are fragmented into smaller parts and 
distributed to many nodes across the network, then executed concurrently. 
In the absence of shared
memory, process synchronization becomes essential. Traditional process
synchronization methods are often too low-level. The message passing routines
are deeply coupled within the program logic. This greatly impacts 
the flexibility of design and the maintainability of the codes.

Recently the Erlang \cite{erlang} style of actor programming is becoming
an interesting case study. In Erlang, inter-process communication
routines are expressed in terms of explicit {\tt send} and {\tt receive}
constructs. Data and messages are passed among nodes via these
constructs. Incorporating this feature into main stream imperative
languages such as C and Java becomes an exciting problem. 

Software Transactional Memory (STM) \cite{stm} was recently introduced to
community. The purpose of using STM is to reduce the effort of 
writing deadlock-free programs. The programmer is freed from 
writing explicit locks and semaphores for synchronization. (This is
similar to the introduction of garbage collection, which 
frees the programmer from writing {\tt malloc} and {\tt free} statements.) Currrently, Software Transactional Memory is only applicable in the context of 
shared memory processors (SMP). It is also possible to apply the
idea to distributed computing settings. For instance, via memcache
we are able to allow multiple nodes on a network to share a common
memory space.  

My idea is to unify the concepts from these two approaches into a
calculus and apply it to main stream languages such as C and Java. 
This would allow us to develop a highly-useful distributed computing framework.
}
\comment{
Recently, there have been some work \cite{cchr} which unifies the ideas of the 
above two projects. Via a notion of Concurrent Constraint Handling
Rules, it is possible to describe these two features in one single 
calculus. Therefore, there is ample of space we can explore the hybrid
approach of the two.
}






\newpage
\bibliographystyle{plain}
\bibliography{main}

\end{document}
