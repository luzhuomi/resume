\documentclass[margin,line]{res}


\oddsidemargin -.5in
\evensidemargin -.5in
\textwidth=6.0in
\itemsep=0in
\parsep=0in

\newenvironment{list1}{
  \begin{list}{\ding{113}}{%
      \setlength{\itemsep}{0in}
      \setlength{\parsep}{0in} \setlength{\parskip}{0in}
      \setlength{\topsep}{0in} \setlength{\partopsep}{0in} 
      \setlength{\leftmargin}{0.17in}}}{\end{list}}
\newenvironment{list2}{
  \begin{list}{$\bullet$}{%
      \setlength{\itemsep}{0in}
      \setlength{\parsep}{0in} \setlength{\parskip}{0in}
      \setlength{\topsep}{0in} \setlength{\partopsep}{0in} 
      \setlength{\leftmargin}{0.2in}}}{\end{list}}


\newcommand{\researchonly}[1]{#1}
%%\newcommand{\researchonly}[1]{}
\newcommand{\longversion}[1]{}
%\newcommand{\longversion}[1]{#1}
\newcommand{\ignore}[1]{}

\begin{document}


\name{Zhuo Ming LU, Kenny \vspace*{.1in}}


\begin{resume}
\section{\sc Contact Information}
\vspace{.05in}
\begin{tabular}{@{}p{1in}p{4in}}
{\it Address:} & 12A Canberra Drive 03-26 Singapore 768095 \\
{\it Phone:} &  (65) 9012-1486 \\            
{\it E-mail:} &  luzhuomi@gmail.com \\         
{\it Homepage:} & http://sites.google.com/site/luzhuomi/\\     
{\it Blog:} & http://luzhuomi.blogspot.com \\
{\it Linkedin:} & http://www.linkedin.com/luzhuomi/  \\
{\it Github:} & http://github.com/luzhuomi/
\end{tabular}

\section{\sc Nationality}
Singaporean


\section{\sc Education} 
\begin{tabular}{ll}
 {\bf Jun 2009}  & {\bf Ph.D. in Computer Science, } \\ 
		       & School of Computing, National University of Singapore \\ 
		       & Topic: XHaskell - Adding Regular Expression Type to Haskell \\ 
		       & Supervisor: Prof. Dr. Martin Sulzmann \\ \\
 {\bf July 2001} & {\bf Bachelor of Science with 2nd Upper Honor,} \\ 
		       & Depart of Computer Science, \\
		       & National University of Singapore \\ 
		       & Topic: Models and Methods of Web portal development \\ 
		       & Supervisor: A/P Wei-Ngan Chin \\ \\
% July 1996  & Secondary school graduate, Foshan Number One Middle School, \\ 
%	    & China \\ \\
% July 1990  & Primary school graduate, Foshan Number Twenty-five Primary \\ 
%            & School, China
\end{tabular}



\researchonly{
\section{\sc Research Interests}
{\bf Programming Languages:} Regular expression and formal
languages. Compiler design. Domain Specific Language. Type
theory. Static analysis.

{\bf Business Analytics and Big Data:} Data mining, information extraction and categorization. Social media analytics.
Generalized data parallelism, Type-safe NoSQL query optimizations.

{\bf Computer Security:} Code obfuscation, Static analysis.

\section{\sc Research Grant}
\begin{enumerate}
\item Android Application Obfuscator. 2019. Capability and Translational Deveopmentt
Grant (RPF02) Amt Awarded: SGD81450
    \begin{itemize}
       \item Role: Chief Project Investigator
       \end{itemize}
\item Collamine: The Collective Framework for Data Mining. 2013. MOE TIF
Grant (MOE2013-TIF-1-G-031) Amt Awarded: SGD239889
    \begin{itemize}
       \item Role: Chief Project Investigator
    \end{itemize}
       
\end{enumerate}
    

%\section{\sc Project}
%MoodSense - a collaboration with the SmartHub project with StarHub Singapore, 2012/2013 
%    \begin{itemize}
%       \item Role: Chief Project Investigator
%    \end{itemize}


\section{\sc Publications}

{\bf Book Chapter}
\begin{enumerate}
\item Lean Launch Data Engineering Projects With Super
    Type Power. \textit{Kenny Zhuo Ming Lu}, To appear in Innovative
    Technologies for Market Leadership: Investing in the
    Future. Editors, \textit{Glauner and Plugmann}, 2019, Springer.
\end{enumerate}



{\bf Journal Publications}
\begin{enumerate}
\item Derivative-Based Diagnosis of Regular Expression
  Ambiguity. \textit{Martin Sulzmann and Kenny Zhuo Ming Lu} IJFCS 2017.
\end{enumerate}


{\bf Conference Publications}
\begin{enumerate}
\item  Solving of Regular Equations Revisited. \textit{Martin Sulzmann
    and Kenny Zhuo Ming Lu}, ICTAC 2019. Hammamet, Tunisia 2019  
\item Control flow obfuscation via CPS transformation. \textit{Kenny
    Zhuo Ming Lu}  PEPM@POPL 2019: 54-60, Cascais, Portugal 2019 
\item Derivative-Based Diagnosis of Regular Expression
  Ambiguity. \textit{Martin Sulzmann and Kenny Zhuo Ming Lu} CIAA 2016.
\item Implementing Cost-effective Data Collection and Extraction
  Processes with CollaMine. \textit{Kenny Zhuo Ming Lu and Belson
    Heng} ICCCRI 2016.
\item POSIX Regular Expression Parsing with Derivatives. \textit{Martin Sulzmann, Kenny Zhuo Ming Lu}  FLOPS 2014  
\item Finding Near Duplicates in Short Text Messages in Singlish Using
  MapReduce and Phonetic Matching. \textit{ Jophia Yi Wen Soh and
    Kenny Zhuo Ming Lu} ICCCRI 2014
\item Regular Expression Sub-Matching using Partial Derivatives. \textit{Martin Sulzmann and Kenny Zhuo Ming Lu} 
In Proceedings of the 14th International Symposium on Principles and Practice of Declarative Programming, PPDP 2012, Leuven, Belgium, September, 2012
\item XHaskell - adding regular expression types to Haskell. \textit{Martin Sulzmann and Kenny Zhuo Ming Lu} 
In Proceedings of the 19th International Symposium on Implementation and Application of Functional Languages, IFL 2007, Freiburg Germany, October 2007.
\item XHaskell - adding regular expression types to Haskell. \textit{Kenny Zhuo Ming Lu} 
In poster session of the Fifth Asian Symposium of Programming Languages and Systems, APLAS 2007,
Singapore. November 29 - December 1 2007.
\item XHaskell - adding regular expression types to Haskell. \textit{Martin Sulzmann and Kenny Zhuo Ming Lu} 
In Proceedings of the 19th International Symposium on Implementation and Application of Functional Languages, IFL 2007, Freiburg Germany, October 2007.
\item XHaskell \textit{Martin Sulzmann and Kenny Zhuo Ming Lu} 
In demo presentation of PLAN-X 2006, Charleston, USA, Jan 2006
\item A Type-Safe Embedding of XDuce into ML. \textit{Martin Sulzmann and Kenny Zhuo Ming Lu} In Proceedings of
the ML Workshop 2005, Tallinn Estonia. Electronic Notes in Computer Science, pages 229-253, 2005 
\item An Implementation of Subtyping among Regular Expression Types. \textit{Kenny Zhuo Ming Lu and Martin Sulzmann} In proceedings of the Second Asian Symposium of Programming Languages and Systems, APLAS 2004, Taipei, Taiwan, November 4-6, 2004. Proceedings. Lecture Notes in Computer Science 3302 Springer 2004.
\end{enumerate}


\longversion{
{\bf Technical Reports (Selected)}
\begin{enumerate}
\item A Faithful Semantics for Hindley/Milner with Regular Expression Types (Extended Version). \textit{Martin Sulzmann and Kenny Zhuo Ming Lu} 2007 July
\item Type Inference and Compilation for Parametric Regular Data Types. \textit{Martin Sulzmann and Kenny Zhuo Ming Lu} 2006
\item A Type and Semantic Preserving Translation from XDuce to ML. \textit{Martin Sulzmann and Kenny Zhuo Ming Lu 2005}
\item An Implementation of Subtyping among Regular Expression Types. \textit{Kenny Zhuo Ming Lu and Martin Sulzmann} NUS Technical Report (TRB9/04)
\item XHaskell: Regular Expression Types for Haskell. \textit{Kenny Zhuo Ming Lu, Martin Sulzmann} NUS Technical Report (TRC9/04) 
\end{enumerate}



\section{\sc Conference Presentation}
\begin{list2}
\item XHaskell: PLAN-X 2006 demo presentation (collocated with POPL 2006)
\end{list2}
}

{\bf Invited Talks}
\begin{enumerate}
 \item Applied Research in Quality Assurance and Data Analytics, 2017 July, Software Quality Chatper, Singapore Computer Society.
 \item Industry Experience - Building A High-Performance Analytics System for the Hospitality Industry, 2013 March, SAS Bites, SAS User Group Singapore.
\end{enumerate}

\section{\sc Patent}
\begin{list2}
\item Method For Information Extraction Failure Diagnosis and
  Auto-recovery (Singapore Patent Application NO. 10201505372X)
\end{list2}
}



\section{\sc Employment History}
\begin{tabular}{ll}
 Mar 2012 - Present & Specialist (AI \& Analytics) and Senior
                                        Lecturer, Nanyang Polytechnic,
                                        \\ & School of Information Technology \\ \\
 Sep 2018 - Present & Honorary Lecturer, University of Glasgow
                       \\ \\   
 Mar 2018 - Mar 2018 & Guest Lecturer, Karlsruhe University of Applied Sciences, Germany \\ \\ 
 Oct 2017 - Oct 2017 & Guest Lecturer, Karlsruhe University of Applied Sciences, Germany \\ \\ 
 Jun 2010 - Feb 2012  & Director of Engineering, Circos Brand Karma \\ \\
 Apr 2008 - Jun 2010  & Senior Software Engineer, Circos Brand Karma \\ \\
 Apr 2006 - Feb 2008,  & Research Assistant, Department of Computer
                         Science, \\ \\
 Aug 2003 - Jul 2004  & School of Computing, National University of Singapore \\ \\
 % August 2004 - March 2006 & Full time research study \\ \\
 % August 2003 - July 2004 & Research Assistant, Department of Computer Science, \\
 %                 & School of Computing, National University of Singapore \\ \\
 August 2001 - July 2003 & Teaching Assistant, Department of Computer Science, \\
	          & School of Computing, National University of Singapore 
\end{tabular}

\section{\sc International Collaboration Experience}
\begin{list2} 
\item  July 2016 - Present: Initiate, setup and manage an exchange program between Nanyang
                       Polytechnic and Karlsruhe University of Applied Sciences.  
\end{list2}

%\longversion{
\section{\sc Recent Teaching Activities}
\begin{list2} 
\item 2019/2020 SEM 1: Data Science Foundation (Nanyang Polytechnic)
\item 2018/2019 SEM 2: Parallel Computing for Big Data (Nanyang Polytechnic)
\item 2018/2019 SEM 1/2: Functional Programming (University of Glasgow)
\item 2018/2019 SEM 2: Enterprise Application Development (Nanyang Polytechnic)
\item 2018/2019 SEM 1: Parallel Computing for Big Data (Nanyang Polytechnic)
\item 2018/2019 SEM 1: Cloud Copmuting (Nanyang Polytechnic)
\item 2018/2019 Spring: Data Analytics and Machine Learning  (Karlsruhe University of Applied Sciences)
\item 2017/2018 Winter : Compiler Design (Karlsruhe University of Applied Sciences)
\item 2017/2018 SEM 2: Enterprise Application Development (Nanyang Polytechnic)
\item 2017/2018 SEM 2: Parallel Copmuting for Big Data (Nanyang Polytechnic)
%\item 2012/2013 SEM 1: IT1209 Interactive Web Design
%\item 2012/2013 SEM 1: IT2592 Enterprise Business Application Project
%\item 2012/2013 SEM 1: IT3180 Emerging Trends and Technologies
%\item 2012/2013 SEM 2: IT3800 Open Source Technologies Development
%\item 2012/2013 SEM 2: IT2126 Object Oriented Programming
%\item 2008/2009 SEM 1: CS4212 Compiler Design (Part time)
%\item 2001/2002 SEM 1: CS2102 Introduction to Database
%\item 2001/2002 SEM 2: CS3214s Software Engineering Projects
%\item 2002/2003 SEM 1: CS3214s Software Engineering Projects
%\item 2002/2003 SEM 2: CS3214 Software Engineering Projects
\end{list2}
%}


\section{\sc Industrial Experience}
%\vspace{-.3cm}
{\bf Circos Brand Karma} 
\\
The mission of Circos Brand Karma is to protect the value and cultivate the potential of 
brands by providing the smartest technologies to analyze data and consultation service
to hospitality industry. 
Circos Brand Karma is the leader in the social media analytic solution market.
Circos Brand Karma the top choice among many global major hotel chains including Starwood, IHG, Mandarin Oriental and Shangri-La.
Brand Karma is the main product of Circos Brand Karma, which is equipped with an
information mining and analytic engine that gathers user reviews and opinions 
across all social media channels. Reviews and experiences
are qualitatively measured by a semantic analyzer.
The results will be applied in brand performance analysis, 
market research and campaign management.

{\bf \em Director of Engineering} \hfill {\bf May 2010 - Feb 2012} \\
As the lead of the engineering team of Circos Brand Karma. My main responsibilities include the following:
\begin{itemize}
  \item Overseeing the product, engineering, quality assurance and customer support teams;
  \item Leading product development and technological innovation;
  \item Managing project schedule and project resource;
  \item Directly reporting to the CTO and the COO;
\end{itemize}
%
Main contribution includes
\begin{itemize}
  \item Reduced data deployment time by 100\%
  % \item Reduced project delays by 50\%
  \item Increased the development team size by 200\%
  \item Delivered three critical produce features
  \item Re-designed backend API sub-system whose performance is increased by factor of 10.
\end{itemize}
%
{\bf \em Senior Software Engineer} \hfill {\bf April 2008 - May 2010} \\
Responsibilities include
\begin{itemize}
  \item Co-leading the development of Brand Karma web application;
  \item Main developer of the core of the Brand Karma API using Haskell and Python;
  \item Reporting to the Director of Engineering;
\end{itemize}
Main contribution includes
\begin{itemize}
  \item Re-engineered the matcher in Haskell, performance increased by 10 times
  \item Developed the first two version of Brand Karma
\end{itemize}
\ignore{
I was in charge of a few components in this products, 
\begin{itemize}
 \item Using Haskell, I developed a system that infers similarity among hotel brand information
   coming from various data sources, (e.g. booking.com and asiaroom.com) 
   in different languages (e.g. English and Chinese). 
   Brand information is often incomplete and described differently in different sources. The system
   identifies similarities among the brands based on the context and (partially complete) attributes. 
   The system is scalable and capable to process gigabytes of unreliable data. 
   The system equipped with its own scheduler, with Haskell STM, it is able to handle concurrent jobs. 
 \item I developed a query parser which parses search queries which are input through
   the ``single text box'' which is a GUI component in www.circos.com pages. The 
   parser takes a string of text and parses it into a set of parse
   results. Each parse result corresponds to one interpretation of the
   input text. They are ranked based on confidence level. The parse
   results will be used in executing the search query. The core of this component is written 
  in Haskell.
 \item I co-led the development of the web based UIs of brandkarma.circos.com which are mainly in python.
\end{itemize}
}

\section{\sc Certification}
{\bf Coursera} Deep Learning Specialization, Stanford University, June
2019 \\
{\bf Coursera} Introduction to Logic, Stanford Univerity, March 2017 \\
{\bf Coursera} Machine Learning, Stanford University, March 2016 \\




\longversion{

\ignore{
\section{\sc Part time and Internship}
{\bf Keppel Shipyard Pte Ltd.} Singapore
%\vspace{-.3cm}
\\
{\em Attachment programmer} \hfill {\bf May 1999 - Nov 1999}\\
Developer of Oracle based in-house project. In charge of front-end window-based "Subcontractor Returning System" and part of back-end database design.
%\vspace{-.3cm}

{\bf CharterTelecom Pte Ltd.} Hong Kong
\\
{\em Part time Technical Engineer} \hfill {\bf Jan 2002 - Apr 2004} \\
In charge of installation, maintenance, configuration of data server, IVR and voice gateway 
(varies models from Unified, Cisco and etc.)
}

\section{\sc Academic Projects} 
{\bf XHaskell }\\
We extend the main stream functional language Haskell with XML processing features from XDuce. 
We combine the advanced languages features from the two, such as regular expression types (XDuce),
regular expression patterns (XDuce), algebraic datatypes (Haskell),
parametric polymorphism (Haskell) and adhoc polymorphism (Haskell).
Such a language extension is highly useful in developing 
type-safe XML transformation, it guarantees that resulting
documents are always valid with respect to the schema specifications.
\\ \\
{\bf PPChugs} \\
{\bf http://code.google.com/p/ppchugs/} \\
I port the Hugs (A Haskell 98 interpreter) to Pocket PC.
The project is developed using MS Visual Studio Embedded.
\\ \\
\section{\sc Part Time Projects} 
{\bf regex-pderiv} \\
{\bf http://hackage.haskell.org/regex-pderiv} \\
An efficient implementation of regular expression pattern matching using partial derivatives in pure Haskell.

{\bf vim for E680} \\
%{\bf PPChugs : http://www.comp.nus.edu.sg/\verb+~+luzm/vim} \\
I port the vim editor to the Motorola phone E680i.
The editor is cross-compiled using gcc tool-chain.
\\ \\
\ignore{
{\bf Eship-biz Web-portal} \\
Applying state-of-art methods and tools to develop a commercialized web portal 
for E-Ship Business Ptd. We developed an online portal to facilitate services like product enquiry, stock update, PO transaction, payment
transaction, delivery tracking. \researchonly{The system is developed on WindowsNT platform using a IIS webserver. Varies
system component are developed using ASP, C++ and Java frameworks. The integration is done via JNI API. The 
unifying stock update component plays an important role in the system, which operates by message/data passing
using XML documents.} }
\\ \\



\section{\sc Computer Technical Skills} 
\begin{list2}
\item Programming Languages:  
      \begin{itemize}
       \item Proficient: Haskell, Erlang, Python,  SQL, 
       \item Fair: CHR, Clojure, Scala, C$\sharp$, Java, XDuce, CDuce, OCaml, SML, Scheme, Prolog, C,  C++, perl, Lisp,  Pascal, Unix shell scripts. \\
      \end{itemize} 
\researchonly{\item Applications:  \LaTeX, common MS word, spreadsheet, and presentation software.  \\}
\item Web Techs: Django, Yesod, Mono, Apache, Lighttpd.  \\
\item Database Servers: Postgres, Oracle, Sybase, Windows SQL Server.  \\
\item NoSQL: MongoDB, memcached, redis. \\
\item Operating Systems:  Mac OSX, Linux, Solaris, FreeBSD, Windows, QNX \\
\item Cloud Computing: Hadoop, Amazon EC2 
\end{list2}





\section{\sc Honors and Awards} 
Dean's List award 1997 (Sem 1/2)
Dean's List award 2000 (Sem 1)


%\section{\sc Academic Results}
%{\bf Postgraduate:} \\ \\
%Advanced Topics in Program Languages \hfill {\bf A+} \\
%Reasoning Under Uncertainty          \hfill {\bf A-} \\
%Distributed System                   \hfill {\bf B+} \\ 
%\\
%{\bf Undergraduate:} 
%\textit{Please refer to attached transcript.} \\

\section{\sc Advanced Computer Knowledge}
\begin{list2}
\researchonly{\item Formal semantics including language dynamic/static semantics; \\}
\item Program Compilation and Optimization; \\
\researchonly{\item Program Analysis \\ }  
\item Functional Programming, Logic Programming, Type system, Pattern Matching; \\
\item Compiler design; \\
\item System design with UML, project planning and management using Microsoft Project or Google docs;
\end{list2}


\section{\sc Other Activities}
\begin{list2}
\item Co-Founder and chairperson of Singapore Functional Programming Group \hfill {\bf 2009-Present}
\end{list2}
\ignore{
\begin{list2}
\item Chairman of Computer Committee in Eusoff Colleague \hfill {\bf 2002}
\end{list2}
}
}

\longversion{
\section{\sc Personal Particular}
\begin{tabular}{@{}p{1in}p{6in}}
Date of Birth: & 04/04/1978 \\ 
Gender: & Male \\
Languages: & Mandarin (Native), Cantonese (Native) and English (Fluent)
\end{tabular}
}


\section{\sc Referees}
\begin{list2}
\item {\bf  Mr George Mitchell} \\
Engineering Manager \\
Riot Games  \\
Email: gogito@gmail.com\\ 
\item {\bf  Prof. Dr. Martin Sulzmann} \\
Hochschule Karlsruhe - Technik und Wirtschaft 
Fakultät für Informatik und Wirtschaftsinformatik  \\
Phone:  +49 721 925-1570  \\
Email: Martin.Sulzmann@hs-karlsruhe.de\\ 
%\item {\bf Associate Professor Dr. Wei-Ngan Chin} \\
%Department of Computer Science, School of Computing, National University of Singapore \\
%S15 Level 6, 3 Science Drive 2 Singapore 117543 Republic of Singapore \\
%\ignore{ Phone: (65)6516-6228 \\
%Fax: (65)6779-4580 \\ }
%Email: chinwn@comp.nus.edu.sg\\
\item {\bf Associate Professor Dr. Siau-Cheng Khoo} \\
Department of Computer Science, School of Computing, National University of Singapore \\
Computing 1, Law Link Singapore 117590 \\
Phone: (65)6516-6730 \\
Fax: (65)6779-4580 \\ 
Email: khoosc@comp.nus.edu.sg\\

\ignore{ 
\item {\bf Mr. Linus Wong} \\
Charter Telecom Pte Ltd, Hong Kong, \\
House 43F, Ping Long Tsuen, Lam Tsuen Taipo \\
Phone: (852)96818219 \\
Email: ong@chartertelecom.com.hk}
\end{list2}


\end{resume}
\end{document}




